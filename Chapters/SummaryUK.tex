\chapter{Summary (English)}

At this point in time, it is especially interesting to research the ability of drones to navigate autonomously, as we live in a world where completely autonomous agents already now are becoming a part of our everyday lives. 

The goal of this thesis is to implement and document a set of different swarm algorithms, and compare them in their ability to control multiple quadcopters, as individual agents. Furthermore, the goal of this thesis is to improve the abilities of the implemented swarm algorithms, based on how they perform in simulated tests.

In this thesis, a simulation model is constructed using the Simulink simulation software. The implementation of this simulation is documented as the first part of the thesis, and the model is used to subject the simulated drones to physical laws such as gravity, to see how well their control algorithms respond.

After the implementation of the simulation model, the set of swarm algorithms are described and later implemented and tested. The results of these tests are used to optimize the implemented swarm algorithms and suggest further optimizations. These optimizations are based on techniques from other swarm algorithms, where some act as the original contribution of this thesis. 

After re-testing different aspects of the swarm algorithm, the effectiveness of the optimizations was found to by limited. In the case of the Stigmergy algorithm, the optimizations can be said to both have pros and cons, as some aspects of the algorithms increased in performance, but others suffered opposite results.

The thesis is concluded with an evaluation of the subproblems of the thesis, and whether or not each of them was given a suffucient answer, both in terms of academic value and expected complexity. Furthermore suggestions for further research are discussed. 
