
\chapter{Methodology}
In order to evaluate the different control algorithms and their performance, the approach towards designing and testing the algorithms must be systematic and concise. 
The different subproblems specified in \ref{intro:sub_problems}, can be said to either relate to either the simulation design or the algorithm design. As such, for the purpose of this thesis, a different approach will be used when dealing with constructing the simulation and when designing and testing the control algorithms. In this chapter, I will outline these approaches and the methodology used. The first three sections of this chapter will deal with the simulation methodology, whereas the latter three will deal with the implementation and testing of the control algorithms. 

\section{Control Problem}
The main research problem os this thesis involves building and evaluating different control algorithms, and as such each algorithm must be subject to the same evaluation technique.
This evaluation technique will throughout the thesis be referenced to as the \textit{Control Problem}, and the purpose of the agents will be so solve this problem collectively. 
As such the problem chosen must be the same problem throughout every simulation and independent of the control algorithms themselves. The problem must also be able to be solved collectively, by multiple agents at the same time, while still posing enough challenges to properly test the control algorithms. Posing dangers to the agents, such as the risk of collision. 

This control problem will exist within the simulation as a physical task that the agents have to carry out. Chapter \ref{sec:control_problem} will outline the concrete control problem used, as well a discussion on the specific control problem's advantages and limitations. 

\section{Simulation Implementation}
The control problem and the simulation are mutually dependent, as the problem has to able to exist and be solvable within the simulation. As such the simulation has to be designed with the control problem in mind, but the control problem also has to be made with the limitations of the simulation taken into account. The simulation aspect is crucial to the validity of the results expressed in this thesis. For the purpose of solving the problems outlined in this thesis, the simulation will be a virtual environment, in which drones can be deployed. The drones will have to be subject to the same physical laws, as they would in real life such that the findings and test results can by applied to issues and control techniques in the real world.

In order to reduce complexity of the simulation, some physical laws are not taken into account when designing the simulation. This is done based on assumptions on which laws influence the results. An outline of the simulation design, can be found in Chapter \ref{chap:simulation}. 

\section{Validation}
Validation is a method to ensure that the simulation is mimicking real life physics and not simply animating the desired behavior of the drones. The validation will be carried out iteratively along with the construction of the simulation, to monitor the progress of the simulation construction, and act as a sort of checklist to the accuracy and capabilities of the simulation. The validation phase consists of multiple validation tests, meant to test if the simulation actually simulates the physical mechanics required. These mechanics can range from gravitational pull to accurate impulse and mass on the bodies in the simulation. The validation tests will when range from checking the accurate acceleration of drones in free fall etc. These validation tests are also important because the simulator must have the predictable behavior across many scenarios, such that they can be compatible. The validation tests will be discussed and outlined in Section \ref{sec:validation}

\section{Test Metrics}
Up until now this chapter on methodology has been focused on systematically creating and validating the simulation platform. The following part will be covering my approach to creating and testing the control algorithms. 

The comparison of the algorithms tested within this thesis, will be done based on their ability to solve the Control Problem. However the control problem must be designed in such a way, that it can be solved to various degrees. To measure how well the control problem is solved, some metrics must be chosen so that they represent the challenges of the control problem. I.e. if the control problem challenges the agents in collision avoidance, a test metric could be each agent's distance the all other agents over time. Time to solve the problem will also act as a simple test metric. The metrics chosen will be covered in Chapter \ref{chap:simulation} and will in part be based on the challenges of swarm robotics covered in \texttt{Multi-robot navigation in formation via sequential convex programming} \cite{alonso-mora_multi-robot_2015}.

\section{Control Implementation}
The algorithms tested within this thesis and presented as \textit{swarm algorithms} will be covered in depth in Chapter \ref{chap:swarm}. When implementing and testing the algorithms in this paper, the algorithms will be partly implemented through experimentation. This is due to the fact that swarm algorithms such as the \textit{Artificial Bee Colony} (or ABC) \cite{bhattacharjee_multi-robot_2011}, cannot directly be mapped to the issue of controlling quadcopters to perform construction tasks. The ABC has a different number of agents and multiple categories. Because of this, mapping some of the algorithms will involve some experimentation and for the sake of methodology it must be stated that algorithms will undergo a level of personalization when implemented. This will be documented along with how much the algorithms ends up differing from its theoretical counterpart. 

\section{Optimization}
In contrast to the Simulation Implementation, the implementation of the control algorithm will not be followed by a validation step in the same way. This is due to the fact the control algorithms cannot be validated with the same fixed criteria, as their desired behavior deviates for each algorithm. Instead, they will undergo optimization. This refers to the activity of not only mapping the control algorithms to control quadcopters, but also tuning them in an attempt to maximize their performance.This will be in benchmarked with the test metrics and the control problem within the simulation. The optimization will only be done on the control algorithms themselves, and not by tuning simulation variables, as all algorithms bust be tested on the same simulation. The simulation variables will be predetermined in the \textit{Validation} phase (see Section \ref{sec:validation}) and hereafter remain static. 



