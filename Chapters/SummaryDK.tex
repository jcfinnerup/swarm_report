\chapter{Summary (Danish)}
\begin{otherlanguage}{danish}

Det er idag specielt interessant at undersøge dronesværmes evne til at navigere selvstændigt og løse problemer sammen, da vi lever i en tid hvor komplet autonome robotter allerede så småt er begyndt at blive en realitet og en praktisk del af menneskers hverdag.  

Målet for denne afhandling er at implementere forskelling sværm algoritmer, og sammenligne dem i deres evne til at kontrollere quadcoptere. Derudover er målet med opgaven at forbedre de implementerede algoritmer, baseret på hvordan de klare sig i en simuleret test. 
I denne afhandling konstrueres først en simulationsmodel i softwaren Simulink, som igennem opgaven benyttes til at simulere droners fysiske egenskaber og måle effektiviteten af deres styringsalgoritme. Implementeringen af simulationen gennemgået nøjagtigt med beskrivelser af de dertilværende overvejelser. 

Efter implementeringen af simulations platformen, beskrives de forskellige sværm algoritmer som senere implementeres og testes i simulationsmodelen. Resultatet af disse tests benyttes herefter til at optimerer de implementerede kontrol algoritmer og foreslå forbedringer ved hjælp af forskellige teknikker. Disse teknikker indeholder elementer fra andre sværm algoritmer men også nye tiltag, der fungerer som denne afhandlings originale bidrag. 

Effektiviteten af de implementerede forbedringer viste sig efter tests i simulationen, at være diskuterbare. Effekten af de forsøgte forbedringer kan i tilfælde af 'Stigmergy' algoritmen, siges både at have fordele og ulemper. 

Afhandlingen afsluttes med en konklusion på de opnåede forbedringer og en diskussion af mulige fremtidige forskningstiltag for yderligere forbedring af sværmalgoritmerne implementeret heri.

\end{otherlanguage}