\chapter{Introduction}

In today's world, we increasingly rely on robotics to perform a wide range of tasks, with use cases ranging from complimenting, or even substituting, human labour in factories to more complex tasks such as automated areal photography. 
Though technology continues to improve, the use of robotics is often confined to within predictable environments, where machines are told what to do and when to do it. 
More recently though, as control algorithms improve, autonomous robots are starting to become a practical reality, with increasing ability self-plan and work unsupervised. 
As such we experience a paradigm shift, where robots are no longer limited to predictable and confined environments, but can act freely and adapt to changing circumstances. 

As a consequence of this, new use cases and areas within robotics are starting to gain interest. 
Specifically the area of autonomous flying vehicles is showing great promise, 

- from supervised to unsupervised
- from instructed to self planning
- from ground-based to arial
\section{Problem Description}




\section{Project plan}
We note that the contents of the project plan is also something we would like to see in the introductory chapter of your thesis. In fact, you can reuse your final project plan (possibly extended) as the introduction. If you prefer to write an introduction from scratch, it is, of course, important that it is consistent with the final project plan.



% \section{The ``separate document''}
% It is also important to note that the separate document containing
% \begin{itemize}
% \item original project plan
% \item possibly revised project plan.
% \item brief self-evaluation
% \end{itemize}
%
% mentioned above will be passed on to the external examiner and since it contains the learning goals and the objectives for your thesis, it will be taken into account when your thesis is assessed.
%
% Awesome awesome awesome

